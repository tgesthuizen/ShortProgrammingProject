\documentclass[a4paper, 11pt]{article}

\title{The ``lyn'' programming language}
\author{Tim Gesthuizen}

\usepackage{tabularx}
\setlength{\parindent}{0ex}
\setlength{\parskip}{1.5ex}

\usepackage{listings}

\begin{document}

\maketitle

\section{Grammar}

Lyn is somewhat Scheme-like.
However, as it has no built-in dynamic memory management at all
constructs that require dynamic memory will not be available.
Most notably, there are no built in data structures, nor function
clojures.

The language consists of the following expressions:\\[1.5ex]
\begin{tabularx}{\linewidth}{lX}
  \texttt{c} & Constant value\\
  \texttt{x} & Variable references\\
  \texttt{(f arg \ldots)} & Function application\\
  \texttt{(lambda (params \ldots) body)} & Abstraction\\
  \texttt{(let ((var val) \ldots) expr \ldots)} & Let expressions:
  Binds \texttt{val}s to the names \texttt{var}s for the duration of
  the \texttt{expr}essions.
  Returns the result of the last expression in the body or the Unit
  value in case there are no body expressions.\\
  \texttt{(if cond expr1 expr2)} & Defines a conditional
  expression, evaluating \texttt{expr1} if \texttt{cond} is
  \texttt{True}, \texttt{expr2} otherwise\\
\end{tabularx}

Types can also be expressed:\\[1.5ex]
\begin{tabularx}{\linewidth}{lX}
  \texttt{int}& Integer type\\
  \texttt{bool}& Bool type\\
  \texttt{unit}& Unit type\\
  \texttt{(-> a1 a2 \ldots ret)} & Function type\\
\end{tabularx}

At the top level the following forms are available:\\[1.5ex]
\begin{tabularx}{\linewidth}{lX}
  \texttt{(define name value)} & Defines a global \texttt{name} to
                                 \texttt{value}.\\
  \texttt{(declare name type)} & Declares a global \texttt{name} with
                                 \texttt{type}.\\
\end{tabularx}

\section{Types}

Every expression and binding has an associated type:\\[1.5ex]
\begin{tabularx}{\linewidth}{lX}
  \texttt{Int} & Integer value\\
  \texttt{Bool} & Boolean Value. Either \texttt{True} or
  \texttt{False}\\
  \texttt{Unit} & Unit value\\
  \texttt{Function(params, res)} & A function taking \texttt{params}
  and returning \texttt{res}\\
\end{tabularx}

Note that, conveniently, all types fit in a single register of the
processor.

\section{Built-in functions}

The following functions are built-in:\\[1.5ex]
\begin{tabularx}{\linewidth}{l|l|X}
  Name & Type & Description \\\hline
  \texttt{+} & \texttt{Int} \rightarrow \texttt{Int} \rightarrow
  \texttt{Int} & Adds two integers\\
  \texttt{-} & \texttt{Int} \rightarrow \texttt{Int} \rightarrow
  \texttt{Int} & Subtracts two integers\\
  \texttt{*} & \texttt{Int} \rightarrow \texttt{Int} \rightarrow
  \texttt{Int} & Multiplies two integers\\
  \texttt{/} & \texttt{Int} \rightarrow \texttt{Int} \rightarrow
  \texttt{Int} & Divides two integers\\
  \texttt{\%} & \texttt{Int} \rightarrow \texttt{Int} \rightarrow
  \texttt{Int} & Takes the remainder of two integers\\
  \texttt{shl} & \texttt{Int} \rightarrow \texttt{Int} \rightarrow
  \texttt{Int} & Shifts an integer to the left\\
  \texttt{shr} & \texttt{Int} \rightarrow \texttt{Int} \rightarrow
  \texttt{Int} & Shifts an integer to the right\\
  \texttt{lor} & \texttt{Int} \rightarrow \texttt{Int} \rightarrow
  \texttt{Int} & Calculates the logical or of two integers\\
  \texttt{land} & \texttt{Int} \rightarrow \texttt{Int} \rightarrow
  \texttt{Int} & Calculates the logical and of two integers\\
  \texttt{lxor} & \texttt{Int} \rightarrow \texttt{Int} \rightarrow
  \texttt{Int} & Calculates the logical xor of two integers\\
  \texttt{neg} & \texttt{Int} \rightarrow \texttt{Int} & Negates
  and integer\\
  \texttt{=} & \texttt{Int} \rightarrow \texttt{Int} \rightarrow
  \texttt{Bool} & Checks if two integers are equal\\
  \texttt{!=} & \texttt{Int} \rightarrow \texttt{Int} \rightarrow
  \texttt{Bool} & Checks if two integers are not equal\\
  \texttt{<} & \texttt{Int} \rightarrow \texttt{Int} \rightarrow
  \texttt{Bool} & Checks if an integer is less than another\\
  \texttt{>} & \texttt{Int} \rightarrow \texttt{Int} \rightarrow
  \texttt{Bool} & Checks if an integer is greater than another\\
  \texttt{<=} & \texttt{Int} \rightarrow \texttt{Int} \rightarrow
  \texttt{Bool} & Checks if an integer is less or equal to another\\
  \texttt{>=} & \texttt{Int} \rightarrow \texttt{Int} \rightarrow
  \texttt{Bool} & Checks if an integer is greater or equal to
  another\\
  \texttt{not} & \texttt{Bool} \rightarrow \texttt{Bool} & Negates a
  boolean\\
  \texttt{or} & \texttt{Bool} \rightarrow \texttt{Bool} \rightarrow
  \texttt{Bool} & \texttt{True} if any of the operands is
  \texttt{True}\\
  \texttt{and} & \texttt{Bool} \rightarrow \texttt{Bool} \rightarrow
  \texttt{Bool} & \texttt{True} if all of the operands are
  \texttt{True}\\
  \texttt{xor} & \texttt{Bool} \rightarrow \texttt{Bool} \rightarrow
  \texttt{Bool} & \texttt{True} if exactly one of the operands is
  \texttt{True}\\
  \texttt{true} & \texttt{Bool} & The true constant\\
  \texttt{false} & \texttt{Bool} & The false constant\\
  \texttt{<>} & \texttt{Unit} & The unit constant\\
\end{tabularx}

\section{Examples}

The Fibonacci function, returning a number of the Fibonacci series by
index.

\lstinputlisting{../examples/fib.scm}

The gcd function, returning the greatest common denominator of two
numbers.

\lstinputlisting{../examples/gcd.scm}

\end{document}